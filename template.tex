\documentclass[14pt,english, margin=1cm]{article}
\usepackage[T1]{fontenc}
\usepackage{babel}
\usepackage{booktabs}
\usepackage{subfig}
\usepackage{float}
\usepackage{graphicx}
\usepackage{titling}
\setlength{\droptitle}{-10em}
\usepackage{dblfloatfix}
\pagestyle{empty}

\usepackage[legalpaper, margin=1cm]{geometry}


\newcommand{\VAR}[1]{}
\newcommand{\BLOCK}[1]{}

\begin{document}
\noindent \textbf{UFMFMS-30-1 TBL Session \VAR{session}}\\

\noindent Weekly report on the engagement of the students on the module UFMFMS-30-1: Dynamics, Modelling and Simulation (DMS).  

\noindent Table~\ref{tab:pre_study_engagement} shows the percentage of students that engaged with the pre-study material. 
Pre-study material include pre-recorded content using the Xerte platform as well as the individual readiness assurance test (iRAT).
Data for the Xerte pre-study material is collected from Blackboard grade-centre and the iRAT data is taken from DEWIS.
\input{table.tex}

\noindent Figure~\ref{fig:trat_performance} shows how groups performed in the team readiness assurance test (tRAT) completed during the lectorial.
Marks are allocated as follows:
\begin{itemize}
\setlength\itemsep{0.1em}
\item 4 marks for question correct with first attempt.
\item 2 marks for question correct with second attempt.
\item 1 mark for question correct with third attempt.
\item 0 marks for non-engagement.
\end{itemize}
On the y-axis of Figure~\ref{fig:trat_performance} is the group numbers and on the x-axis is the possible marks.
The interval of the x-axis is set to the mark allocations of 0, 1, 2 and 4 to improve the readability of the graph.
The bars of Figure~\ref{fig:trat_performance} are color-coded and correspond to a question in the tRAT.
Question 1 is red, question 2 is blue and question 3 is purple.

\begin{figure}[tbh!]
	\centering
	\subfloat[Lectorial group 1.]{\includegraphics[width=0.49\textwidth]{trat_session3_lectorial1_results}\label{fig:lectorial1}} \hfill 
	\subfloat[Lectorial group 2.]{\includegraphics[width=0.49\textwidth]{trat_session3_lectorial2_results}\label{fig:lectorial2}}\\
	\subfloat[Lectorial group 3.]{\includegraphics[width=0.49\textwidth]{trat_session3_lectorial3_results}\label{fig:lectorial3}} 
	\caption{Group tRAT performance.}
	\label{fig:trat_performance}
\end{figure}

\end{document}
